\documentclass[11pt]{article}
\usepackage[left=1in, right=1in, top=1in, bottom=1in, includefoot]{geometry}		% Set margins to 1 inch.

% Reduce the vertical space above the title.
\usepackage{titling}
\setlength{\droptitle}{-8ex}

\usepackage{amsmath}				% For \text{}

\usepackage{amsthm}				% For proof environment
\usepackage{cleveref}
\newtheorem{lemma}{Lemma}
\newtheorem{theorem}{Theorem}

\usepackage{amsfonts}				% For \mathbb{}

\usepackage{graphicx}
\usepackage{floatrow}

\usepackage{nccmath}				% For fleqn environment

\usepackage{color}
\usepackage{xcolor}

\usepackage{bm}					% For \bm{}
\usepackage{physics}				% For \norm{}

\usepackage{tabularx}				% For making tables fit within page width
\usepackage{url}					% For \url{}

\newcommand{\colvect}[2]{
	\ensuremath{\big[\begin{smallmatrix}#1\\#2\end{smallmatrix}\big]}
}

\newcommand{\longtabletext}[1]{
	\parbox[c]{\hsize}{\vspace*{1mm} #1 \vspace*{1mm}}
}

% The \vspace command inside the \title command is used to reduce
% the space between the title and the author lines.
\title{Explaining Predictions of Deep Neural Networks \\ on the Yelp Dataset}
\author{
	Caleb Kaiji Lu \\
	{\tt caleb.lu@andrew.cmu.edu}
	\and
	Selva Samuel \\
	{\tt ssamuel@andrew.cmu.edu}
}
\date{}

\begin{document}

\maketitle

\section{Introduction}

\subsection{Motivation}

In recent years, deep neural networks have been proposed to improve the effectiveness of neural networks in various machine learning tasks such as image classification, natural language processing, speech recognition, etc. However, the models learnt by deep learning systems tend to be very complex and are not easily understandable by humans. This can be problematic when deep neural networks are used to make decisions that have a societal impact such as hiring, credit decisions, prison sentencing, etc. In such applications, it is necessary to check that the decision was made in a fair manner. For example, if an algorithm decides that a job offer should not be made then this decision should have been based on the basis of the candidate's lack of required skill and not on the basis of their gender or ethnicity.

\subsection{Problem Statement}

In this project, we are planning to generate explanations for predictions made by a recurrent neural network that has been trained on the Yelp dataset \cite{Yelp2017}.

\subsubsection{Explaining ratings prediction made on the basis of review text}

Yelp reviews are utilized by users to make decisions about which service provider (restaurants, doctors, etc.) to use for their needs. In making these decisions, users need to understand what aspects of the service provided is reflected in a reviewer's overall rating. For example, if a user is interested in a restaurant which has low ratings, they might need to understand if the low rating is because of bad food or some other aspect (such as ambience, etc.) which they do not care about. So, it is essential to enumerate the factors that contributed to a given reviewer's rating.

\subsubsection{Explaining gender prediction made on the basis of review text}

We are also planning to investigate if the gender of a user can be predicted based on the reviews they have written. If we are able to do that, it would again be important to understand which factors of the review text contributed toward making an accurate prediction. This would provide an understanding of what needs to be done in case a user wants to provide an anonymous review.

\section{Approach}

\subsection{Processing the data and setting up the deep learning system}
\label{step:rating-prediction}

The first step is to set up the recurrent neural network infrastructure \cite{Tang2015} that predicts the ratings from the review text data. We will be reviewing the literature to learn about state-of-the-art deep learning systems that have been used to perform sentiment analysis as well as the prediction tasks that have been accomplished with the Yelp dataset. We will implement these systems and pick the one which gives the best prediction results. Our prediction will be based on either a binary classification problem (which we aggregate the 5-star ratings into 2 groups: positive and negative), or a 5-class classification problem(which we use the 5-star ratings output as it is). 

\subsection{Explaining the deep learning system using functional methods}
\label{step:rating-prediction-explanation}

Several functional methods have been proposed in the literature to explain machine learning models. Some methods such as LIME \cite{Ribeiro2016} and QII \cite{Datta2017} are model agnostic. There are also methods that target deep learning models such as integrated ingredients \cite{Sundararajan2017}. We will implement and compare these functional methods and see which of these methods provide better explanations of the model. Based on the understanding that we gain of the explanation methods and the recurrent neural network, we will try to modify these methods to better comply with the specific structure of recurrent neural network and the specific problem of sentiment analysis.

\subsection{Modifying the deep learning system for prediction of user gender}
\label{step:gender-prediction}

Given the flexible nature of recurrent neural networks, we will change the output layer from ratings to users' gender and adapt the system from step \ref{step:rating-prediction} into a system that predicts users' gender from review text. Since predicting gender is likely to require more text from each user, we will select the users that have written most reviews and aggregate them to see if the system can correctly predict their gender. If so, we will find out what amount of text is needed to correctly identify the gender.

\subsection{Explaining the user gender prediction and addressing the privacy concern}

If step \ref{step:gender-prediction} is successful, we will try to explain the system in the same fashion as in step \ref{step:rating-prediction-explanation}, but with user gender as output. Also, we will try to address this privacy concern by adding a noise to the review or limiting the number of words for the review text and see if that would make a difference in prediction accuracy. 

\section{Deliverables}

The main goal for our project to to build an interactive system such that given a particular review, or group of reviews, or the gender of a user, the system can explain why the deep learning algorithm gives such a prediction. We will provide a color code on the words in a review according to their importance in prediction. We will implement such a system for both rating prediction and gender prediction. We will complete section 2.1 and 2.2 in the approach section for deliverable I, section 2.3 and 2.4 for deliverable II. 

\section{Timeline}

\begin{center}
	\begin{tabularx}{\textwidth}{| l | X |}
	\hline
	\textbf{Week} & \longtabletext{\textbf{Goals}} \\
	\hline
	Sep 25 - Oct 1 & \longtabletext{Refine the scope of the project based on feedback given on the class presentation and review the literature on RNNs} \\
	\hline
	Oct 2 - Oct 8 & \longtabletext{Review the literature on model interpretability measures and setup the RNN infrastructure} \\
	\hline
	Oct 9 - Oct 15 & \longtabletext{Train RNN with Yelp dataset to predict ratings from reviews} \\
	\hline
	Oct 16 - Oct 22 & \longtabletext{Apply functional methods on the trained RNN to explain rating prediction} \\
	\hline
	Oct 23 - Oct 29 & \longtabletext{Finalize Deliverables I} \\
	\hline
	Oct 30 - Nov 5 & \longtabletext{Adapt the RNN for user gender prediction} \\
	\hline
	Nov 6 - Nov 12 & \longtabletext{Train the RNN for user gender prediction} \\
	\hline
	Nov 19 - Nov 26 & \longtabletext{Generate explanation for user gender prediction and address privacy concern by perturbing the review text} \\
	\hline
	Nov 27 - Dec 3 & \longtabletext{Finalize results and write report} \\
	\hline
	Dec 4 & \longtabletext{Presentation and report due} \\
	\hline
	\end{tabularx}
\end{center}

\bibliographystyle{abbrv}
\bibliography{references}

\end{document}
